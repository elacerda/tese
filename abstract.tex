%%%%%%%%%%%%%%%%%%%%%%%%%%%%%%%%%%%%%%%%%%%%%%%%%%%%%%%%%%%%%%%%%
% Tese de Doutorado / Dept. Fisica, CFM, UFSC                   %
% Lacerda@São José - Fev/2017                                   %
%%%%%%%%%%%%%%%%%%%%%%%%%%%%%%%%%%%%%%%%%%%%%%%%%%%%%%%%%%%%%%%%%


%***************************************************************%
%                                                               %
%                          Abstract                             %
%                                                               %
%***************************************************************%

\begin{abstract}

In this thesis we study the diffuse ionized gas (DIG) in galaxies using a spatially resolved sample of 391 galaxies of different Hubble types from the Calar Alto Legacy Integral Field Area (CALIFA) survey. We used the equivalent width of \Ha ($W_{\Ha}$) to separate star-forming regions (SF) from those better characterized as DIG. Three nebular regimes are identified. Regions with $W_{\Ha} < 3$ \AA\ define what we call hDIG, the DIG component where the photoionization is dominated by hot, low-mass, evolved stars. Those with $W_{\Ha} > 14$ \AA\ are identified as star-forming complexes (SFc). Intermediate values of $W_{\Ha}$ reflect a mixed regime (mDIG) where more than one process contributes. This three-tier scheme is inspired both by theoretical and empirical considerations. This scheme application on galaxies of different Hubble-type and inclination lead to the following results:
\begin{enumerate*}[label=(\roman*)]
  \item the hDIG component is prevalent throughout ellipticals and S0's as well as in bulges, and explains the strongly bimodal distribution of $W_{\Ha}$ both among and within galaxies;
  \item early-type spirals have some hDIG in their discs, but this component becomes progressively less relevant for later Hubble types;
  \item hDIG emission is also present above and below galactic discs, as  seen in several edge-on spirals in our sample.
  \item The SF/mDIG proportion grows steadily from early- to late-type spirals, and from inner to outer radii.
  \item Besides circumventing basic inconsistencies in conventional DIG/SF separation criteria based on the \Ha\ surface brightness, our $W_{\Ha}$-based method produces results in agreement with a classical excitation diagram analysis.
\end{enumerate*}

Besides this work, in the appendices I describe some tests with CALIFA spectra to the second CALIFA data-release quality control. Also, I present a module (\emldc) that include the nebular measurements to our analisys pipeline. As a first try, I study the star-formation rate (SFR) and the extintion coefficient. For the SFR, the conclusion is that the synthetic SFR and the one coming from the \Ha luminosity correlate best when the averaged age from the synthetic SFR is 31.62 million years. The study comparing the extinction coefficient calculed from the Balmer decrement and that from the stellar continuum points to a differential extinction scenarium, where younger regiuns are more dust extincted.

\end{abstract}

% End of abstract
