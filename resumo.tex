%%%%%%%%%%%%%%%%%%%%%%%%%%%%%%%%%%%%%%%%%%%%%%%%%%%%%%%%%%%%%%%%%
% Tese de Doutorado / Dept. Fisica, CFM, UFSC                   %
% Lacerda@São José - Fev/2017                                   %
%%%%%%%%%%%%%%%%%%%%%%%%%%%%%%%%%%%%%%%%%%%%%%%%%%%%%%%%%%%%%%%%%


%***************************************************************%
%                                                               %
%                          Resumo                               %
%                                                               %
%***************************************************************%

\begin{abstract}[Resumo]

Nesta tese estudamos o gás ionizado difuso (DIG) em galáxias utilizando uma amostra espacialmente resolvida de 391 galáxias de diferentes tipos de Hubble do Calar Alto {\em Legacy Integral Field Area} (CALIFA) {\em survey}. Nós utiliziamos a largura equivalente de \Ha, $W_{\Ha}$, para separar regiões formadoras de estrelas (SF) daquelas melhores caracterizadas por DIG. Três regimes diferentes foram idenficados. Regiões onde $W_{\Ha} < 3$ \AA\ definem o que chamamos de hDIG, componente do DIG onde a fotoionização é dominada por estrelas quentes e evoluídas de baixa massa. Aquelas com $W_{\Ha} > 14$ \AA\ são identificadas como complexos de formação estelar (SFc). Valores intermediários de $W_{\Ha}$ refletem um regime misto (mDIG) onde mais de um processo contribui. Este esquema com três classes é inspiranda em considerações tanto teóricas quanto empíricas. A aplicação desse esquema nas galáxias de diferentes tipos de Hubble e inclinações leva aos seguintes resultados:
\begin{enumerate*}[label=(\roman*)]
  \item a componente hDIG é prevalente entre galáxias elípticas, S0 e também em bojos, além de explicar a distribuição bimodal de $W_{\Ha}$, tanto ao longo quanto entre galáxias;
  \item galáxias espirais {\em early-type} possuem algum hDIG em seus discos mas essa componente se torna progressivamente menos relevante para tipos de Hubble mais tardios;
  \item emissão hDIG também é presente fora do disco galáctico (acima e abaixo) como foi visto em vários exemplos de galáxias espirais {\em edge-on} em nossa amostra;
  \item a proporção SF/mDIG cresce firmemente ao irmos de espirais {\em early-} até {late-type} e de dentro para fora em galáxias;
  \item além de contornar inconsistências básicas do critério de separação SF/DIG baseado no brilho superficial de \Ha, nosso método baseado em $W_{\Ha}$ produz resultados de acordo com análise com diagramas clássicos de excitação.
\end{enumerate*}

Além desse trabalho, nos apêndices descrevo alguns testes feitos com os espectros do CALIFA para o controle de qualidade do segundo {\em data-release} do CALIFA. Também apresento um módulo (\emldc) que incluiu as medidas nebulares em nossa {\em pipeline} de análise. Como primeira utilização, estudo a taxa de formação estelar (SFR) e o coeficiente de extinção. Para a SFR, a conclusão é que a melhor correlação entre SFR da síntese e aquela proveniente da luminosidade de \Ha acontece quando calculamos a SFR da síntese dos últimos 31.62 milhões de anos. Já o estudo comparando os coeficientes de extinção provenientes do decremento de Balmer e do contínuo estelar aponta para um cenário de extinção diferencial, no qual as regiões mais jovens são mais extintas por poeira.

\end{abstract}
% End of resumo
