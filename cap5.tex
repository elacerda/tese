%%%%%%%%%%%%%%%%%%%%%%%%%%%%%%%%%%%%%%%%%%%%%%%%%%%%%%%%%%%%%%%%%
% Qualificacao de Doutorado / Dept Fisica, CFM, UFSC            %
% Eduardo@UFSC - 2015                                           %
%%%%%%%%%%%%%%%%%%%%%%%%%%%%%%%%%%%%%%%%%%%%%%%%%%%%%%%%%%%%%%%%%


%:::::::::::::::::::::::::::::::::::::::::::::::::::::::::::::::%
%                                                               %
%                          Capítulo 5                           %
%                                                               %
%:::::::::::::::::::::::::::::::::::::::::::::::::::::::::::::::%

%***************************************************************%
%                                                               %
%                     Diferential extintion                     %
%                                                               %
%***************************************************************%

\chapter{Extinção diferencial}
\label{sec:difextin}

O objetivo deste trabalho é conseguir estimar a fração de gás para nossas galáxias do \CAL através
da conversão de poeira em gás explorando uma conversão do tipo da Eq. \ref{eq:dust2gas}. Para isso
vamos tentar entender melhor a diferença entre os coeficientes de extinção por poeira adotados,
realizando um procedimento parecido e fazendo algumas comparações com um estudo de nosso grupo de
populações estelares, que ainda está para ser publicado, sobre extinção diferencial em galáxias do
\SDSS. Este estudo sobre extinção diferencial nas galáxias do \SDSS é parte da tese de Marielli de
Souza Schlickmann.

%\section{Definindo nossas variáveis}
\section{Estudo empírico}
\label{sec:difextin:emp}

Para explorar as diferenças entre os coeficientes de extinção nebular e estelar, identificamos duas
variáveis que nos ajudarão nesse processo:
\begin{eqnarray}
	\mathcal{D}_\tau\ &\equiv&\ \tauVN - \tauVS \\
	\label{eq:difftau}
	\mathcal{R}_\tau\ &\equiv&\ \frac{\tauVN}{\tauVS} 
	\label{eq:ratiotau}
\end{eqnarray}

\subsection{Comparação direta entre os coeficientes}
\label{sec:difextin:emp:comparetauV}

Neste capítulo (e no seguinte) nos atemos em uma discussão utilizando perfis radiais. Na Fig.
\ref{fig:tauVhisto} vemos a comparação direta entre $\tauVS$ e $\tauVN$, juntamente com os
histogramas de $\Dtau$ e $\Rtau$. O primeiro painel mostra a comparação entre os coeficientes como
na Fig. \ref{fig:tauVsynvsnebMask} para perfis radiais. Os dois histogramas são de $\Dtau$ e
$\Rtau$. Os valores da mediana (média) de $\Dtau$ são 0.31 (0.32) e de $\Rtau$, 2.06 (2.46) e ambos
mostram claramente a existência de uma extinção diferencial. Esse resultado aparece da mesma forma
para zonas e para as galáxias integradas. No trabalho de Marielli (com 107231 espectros de galáxias
integradas do \SDSS-DR7) cabe lembrar que a síntese de populações estelares foi feita apenas para
espectros integrados e com diferentes bases de populações estelares e IMF, além do intervalo de
normalização ser feito em uma janela diferente, portanto, alterando também o resultado de $x_Y$;
ainda sim encontramos valores que evidenciam o mesmo fenômeno e que servem de base para a discussão
que segue.

\begin{figure}
	\centering
	\includegraphics[width=0.99\textwidth]{figuras/histoTauVR.pdf}
	\caption[Comparação $\tauVS$ e histogramas de $\Dtau$ e $\Rtau$]
	{\emph{Painel esquerdo}: Igual ao painel superior direito da Fig. \ref{fig:tauVsynvsnebMask}.
\emph{Painel central e direito}: histogramas normalizado de $\Dtau$ e $\Rtau$. Os valores no canto
superior direito marcam a média, mediana, desvio padrão, máximo e mínimo das distribuições.}
	\label{fig:tauVhisto}
\end{figure}

\subsection{O papel de $x_Y$ e de $b/a$}
\label{sec:difextin:emp:xYcosi}

Pela relação entre $\tauVS$ e $\tauVN$ vemos que há um espalhamento notável, entretanto, parece óbvio
que a extinção diferencial existe. Ao estudar algumas propriedades, galáxia por galáxia, vimos que
existe uma relação interessante entre as variáveis que criamos aqui ($\Dtau$ e $\Rtau$), a fração
de populações jovens ($x_Y$) e também a relação axial da galáxia ($b/a$). A primeira pela fato que
relaciona regiões onde existam populações mais jovens, ou seja, mais relacionadas a regiões
formadoras de estrelas, ajudando-nos a quebrar um pouco essa ambiguidade nos valores de $\tauVS$ e
a última pelo fato de afetar diretamente a forma de modelar a extinção por poeira.

Para utilizar a relação axial vamos fazer uma pequena correção para a projeção do disco no plano do
céu:
\begin{equation}
	\varphi = \cos i = \left \{ \begin{matrix} \sqrt{\frac{(b/a)^2 - (0.13)^2}{1 - (0.13)^2}},
	&\mbox{se }b/a > 0.13 \\ 0.05, &\mbox{se }b/a <= 0.13 \end{matrix}\right
\end{equation}
\noindent onde $i$ é o ângulo de inclinação da galáxia e $0.13$ é o valor da relação
axial intrinseca de uma galáxia mais de lado possível ({\em edge-on}). Valores com $b/a$ <= 0.13
recebem o valor da menor inclinação de nossa amostra (0.05) ao invés de 0 para não haver problemas
matemáticos futuros. Esse é um refinamento que altera levemente o valor de $b/a$ e praticamente só
afeta as galáxias com $b/a$ < 0.4, ou seja, na prática $\cos i \sim b/a$.

Na Fig. \ref{fig:Dtau} vemos como $\Dtau$ se relaciona com $\varphi$ e $x_Y$. Nos painéis $a$ e $c$
vemos as distribuições de pontos nos planos \Dtau-($x_Y$,$\phi$), e nos $b$ e $d$ vemos as medianas
para diferentes {\em bins} da variável cruzada, isto é, distribuição contra $x_Y$ com os pontos
divididos em {\em bins} de $\varphi$ e distribuição contra $\varphi$ em {\em bins} de $x_Y$. É clara
a tendência de $\Dtau\ \to\ 0$ quando $x_Y \to 1$. Pelo painel $a$ vemos que apesar da existência de
uma anticorrelação entre $\Dtau$ e $x_Y$, existe um espalhamento considerável. Dividimos nossa
amostra em {\em bins} de $\varphi$ e parece não haver dependência (painel $b$). Nos painéis $c$ e
$d$ exploramos a relação $\Dtau$ e $\varphi$. $\Dtau$ não parece haver dependência com $\varphi$ mas
quando dividimos a amostra em {\em bins} de $x_y$ vemos que se confirma a anticorrelação.

Repetimos esse mesmo experimento para $\Rtau$ na Fig. \ref{fig:Rtau}. Observando a mediana parece
que $\Rtau \to 1$ quando $x_Y \to 1$, com uma $\Rtau \to 3 \sim 4$ quando $x_Y \to 0$. Vemos pelo
painel $d$ que a anticorrelação com $x_Y$ segue, mas olhando o $b$ vemos que parece haver um pequeno
afastamento entre as medianas da distribuição quando dividimos nessas {\em bins} de $\varphi$. Esse
resultado não é evidente quando olhamos o painel $c$. Quando fazemos o mesmo gráfico para as
galáxias do \SDSS esse resultado é muito mais evidente (Figs. \ref{fig:DtauSDSS} e
\ref{fig:RtauSDSS}), aparecendo uma correlação entre $\Rtau$. Em primeira estância acreditamos que
isso se deve ao poder da estatística dos dados do \SDSS. Nosso estudo com valores integrados peca
por estatística, pois temos apenas 184 pontos.

\begin{figure}
	\centering
	\includegraphics[width=0.99\textwidth]{figuras/DtauR.pdf}
	\caption[$\Dtau$, $x_Y$ e $\varphi$]
	{\emph{Painel a}: $\Dtau$ versus $x_Y$ para nossa amostra. Os contornos definem
$1\sigma$, $2\sigma$ e $3\sigma$ da distribuição. As linhas marcam a mediana (linha contínua) e os
5,16,64,95 percentis (linhas tracejadas). \emph{Painel b}: as medianas da mesma distribuição
quando dividimos nossa amostra em {\em bins} de $\varphi$. Os intervalos são criados para que contenham
praticamente o mesmo número de pontos. \emph{Painel c}: igual ao $a$ mas versos
$\varphi$. \emph{Painel d}: igual ao $b$ mas com a amostra dividida em {\em bins} de $x_Y$.}
	\label{fig:Dtau}
\end{figure}

\begin{figure}
	\centering
	\includegraphics[width=0.99\textwidth]{figuras/RtauR.pdf}
	\caption[$\Rtau$, $x_Y$ e $\varphi$]
	{Igual a Fig. \ref{fig:Dtau} mas com $\Rtau$.}
	\label{fig:Rtau}
\end{figure}

\begin{figure}
	\centering
	\includegraphics[bb= 30 370 550 550, width=\textwidth]{figuras/P12fig3.eps}
	\caption[$\Dtau$, $x_Y$ e $\varphi$ no \SDSS]
	{Igual a Fig. \ref{fig:Dtau} mas para os dados do \SDSS.}
	\label{fig:DtauSDSS}
\end{figure}

\begin{figure}
	\centering
	\includegraphics[bb= 30 370 550 550, width=\textwidth]{figuras/P12fig4.eps}
	\caption[$\Rtau$, $x_Y$ e $\varphi$ no \SDSS]
	{Igual a Fig. \ref{fig:Rtau} mas para os dados do \SDSS.}
	\label{fig:RtauSDSS}
\end{figure}

\section{Modelagem e interpretação}
\label{sec:difextin:modeleinterp}

Com esse cenário criado até agora neste capítulo, parece ser óbvio que $x_Y$ é fundamental em nossa
interpretação da extinção diferencial. O resultado da correlação entre $\Rtau$ e $\varphi$
não parece ser tão forte localmente quanto para valores integrados pois ele aparece muito melhor
quando olhamos para a amostra do \SDSS. Seguiremos aqui assumindo essa na linha de interpretação do
trabalho de extinção diferencial nas galáxias do \SDSS. Essa diferença esperamos discutir em um
futuro próximo e entra na lista de coisas para fazer, portando nesta seção vou apresentar o modelo
proposto na tese de Marielli e utilizar a mesma modelagem com nossos dados.

\subsection{O modelo}
\label{sec:difextin:modeleinterp:model}

Segundo tudo que vimos até agora, nosso modelo deve levar em conta que a luz proveniente das
populações jovens são atenuadas de maneira mais intensa que o meio interestelar. Além disso, deve
levar em conta que existe algum tipo de relação entre a razão entre os coeficientes seja afetada
diretamente pelo ângulo de inclinação da galáxia. Apesar de aqui estarmos utilizando $t_{SF} = 32$
milhões de anos, tanto na tese de Marielli quanto no artigo de \citet{Charlot.Fall.2000a} são
consideradas jovens todas as populações com idade menor que 10 milhões de anos.

Podemos ver uma representação visual do modelo na Fig. \ref{fig:model}. A interpretação da
influência de $\varphi$ foi relacionada a diferença entre geometria das regiões \Hii e a da galáxia.
As estrelas jovens, assim como as velhas, possuem atenuações semelhantes a do ISM dos discos
galáticos, mas as populações jovens possuem também a poeira de suas {\em birth clouds}. Ou seja, se
modelarmos as regiões \Hii como esfericamente simétricas, o coeficiente de extinção não deve se
alterar quando mudamos a linha de visada entre o observador e a galáxia. Já para o ISM parece se
alterar bastante, portanto a razão entre os coeficientes deve se alterar no mesmo passo que a
inclinação se altera.

\begin{figure}
	\includegraphics[width=0.99\textwidth]{figuras/Hiiregion_model.pdf}
	\caption[Modelo para extinção diferencial]
	{Figura mostrando o modelo criado. As setas apontam para o observador. São indicados também o
ângulo de inclinação da galáxia, assim como as regiões que delimitam $\tauBC$ e $\tauISM$. Estrelas
azuis significam estrelas jovens e vermelhas estrelas velhas. As camadas de poeira estão pintadas
em cinza.}
	\label{fig:model}
\end{figure}

Essa figura juntamente com as considerações acima nos levam a a identificar um coeficiente de
extinção para as populações jovens e outro para as velhas da seguinte maneira:
\begin{eqnarray}
	\tau_O &=& \frac{\tau_{\mathrm{ISM}}}{\varphi} \\
	\label{eq:tauO}
	\tau_Y &=& \frac{\tau_{\mathrm{ISM}}}{\varphi} + \alpha\tau_{\mathrm{BC}}
	\label{eq:tauY} 
\end{eqnarray}
\noindent onde $\alpha$ entra para corrigir as possíveis diferenças entre a extinção para as
populações jovens e para as suas regiões \Hii associadas.

Para relacionar essas variávies com nossos observáveis, começamos identificando $\tau_Y$ como
$\tauVN$. Como comentamos na Sec. \ref{sec:synvsneb:tauv} $x_Y$ parece funcionar distribuindo os
pesos entre $\tau_O$ e $\tau_Y$ na interpretação de $\tauVS$, assim, podemos aproximar $\tauVS$
como:
\begin{equation}
	\tauVS \approx \tilde{\tau} \equiv x_Y \tau_Y + (1 - x_Y) \tau_O.
	\label{eq:tauVS}
\end{equation}
Com essa última equação fechamos nosso sistema e podemos escrever equações para $\tauISM$ e
$\tauBC$, além de trazer uma interpretação para o real significado de $\tauVS$. Trabalhando com
\eqref{eq:tauO}, \eqref{eq:tauY} e \eqref{eq:tauVS} e identificando $\tau_Y = \tauVN$ obtemos:
\begin{eqnarray}
	\tauBC &=& \frac{\tauVS - \tauVN}{1 - \alpha x_Y} \\
	\label{eq:tauBC}
	\tauISM &=& \varphi\frac{\tauVS - \alpha x_Y \tauVN}{1 - \alpha x_Y}  
	\label{eq:tauISM}
\end{eqnarray}

Pelo histogram dessas variávis (Fig. \ref{fig:tauBCISM}) vemos que o cenário de extinção diferencial
persiste, com valores característicos (mediana) para $\tauBC$ de 0.39 e de $\tauISM$ 0.08. A razão
entre as medianas (médias) resulta em 4.9 (3.8). A mediana da razão nos leva a um valor um pouco
menor (3.43) mas há espalhamento muito grande nesta variável (devido aos valores de $\tauBC$ e
$\tauISM$ que passam por zero). 

\begin{figure}
	\centering
	\includegraphics[width=0.99\textwidth]{figuras/tauISMBC_R.pdf}
	\caption[Histogramas de $\tauBC$, $\tauISM$ e $\rho$]
	{Nos três painéis vemos os histogramas de $\tauBC$, $\tauISM$ e $\rho$. Os intervalos se estendem
para fora do gráfico, portanto, a última classe de cada histograma (em ambos os lados) contém todos
os pontos que ficaram para fora. Os valores nos cantos são os mesmos encontrados em todos os
histogramas deste trabalho: média, mediana, desvio padrão, máximo e mínimo.}
	\label{fig:tauBCISM}
\end{figure}

\section{Próximos passos}
\label{sec:difextin:nextsteps}

Dentre todas as coisas que temos para fazer, uma das principais é comparar este modelo com os dados
para posteriormente podermos discutir as diferenças entre a aplicação deste modelo de extinção
diferencial para galáxias integradas e para perfis radiais (e consequentemente, para zonas também).
Ao estudar as populações estelares quando observamos uma região de uma galáxia, ao invés de uma
galáxia inteira, estamos olhando diferentes estágios de evolução química e formação estelar,
inclusive com uma diferença muito maior entre os coeficientes de extinção dentro da mesma galáxia.
As galáxias do \CAL estão dentro do mesmo intervalo de redshift de forma que todas aproveitem o
campo inteiro de observação, dessa forma, zonas possuem área típica de 1kpc${}^2$. Quando temos
valores integrados, estamos olhando valores que representam uma galáxia inteira (portanto uma
``soma'' de tudo o que acontece no bojo e no disco).


% End of this chapter
