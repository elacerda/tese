%%%%%%%%%%%%%%%%%%%%%%%%%%%%%%%%%%%%%%%%%%%%%%%%%%%%%%%%%%%%%%%%%
% Tese de Doutorado / Dept. Fisica, CFM, UFSC                   %
% Lacerda@São José - Fev/2017                                   %
%%%%%%%%%%%%%%%%%%%%%%%%%%%%%%%%%%%%%%%%%%%%%%%%%%%%%%%%%%%%%%%%%


%***************************************************************%
%                                                               %
%                        Agradecimentos                         %
%                                                               %
%***************************************************************%

\chapter*{Agradecimentos}

Inicio meus agradecimentos pela minha Mãe. Mão forte que está ao lado em todos os momentos desde sempre. Meu norte na vida! Ao meu pai {\em in memoriam} por ser aquele que sempre me instigou na direção da ciência! Meus irmãos, cunhad@s e ao novo sol de nossa família, Iara. À minha família inteira, avós, ti@s e prim@s, minha base, meus amores.

Ao meu orientador, Cid, e nossas conversas regadas a muito vinho e ciência (às vezes). O Cid sempre me ensinou a direção mais divertida da ciência, com as pessoas mais divertidas e, ao mesmo tempo curiosas, que já conheci. À Natalia, que como o Cid, me acompanhou e auxiliou durante esses quatro anos. Obrigado Claudio, Christophe, Rosa, Rubén, Enrique, Sebastián, Rafa, Clara, Mary, Jonnathan {\em y todos los amigos de España, Mexico y Venezuela}.

Um parágrafo especial à Grazyna Stasi{\' n}ska, a cientista mais curiosa que conheci em toda a vida. Grazyna, muito obrigado por tudo, pela amizade e carinho que você sempre me dedicou. E que venham mais e mais pesquisas (... digo {\em fiestas}).

Aos amigos que fiz por toda a vida, todos!

Esse trabalho não seria possível sem todas as instituições que me ajudaram: Universidade Federal de Santa Catarina, UFSC; Programa de Pós-Graduação em Física; PPGFSC; Instituto de Astrofísica de Andalucía, IAA/CSIC. Coordenação de Aperfeiçoamento de Pessoal do Nível Superior, CAPES; Conselho Nacional de Desenvolvimento Científico e Tecnológico, CNPq.

Por último, para Fáfa e Teco, Digão e Rafa, e, em especial, para as duas mulheres de minha vida, que tornam meus dias mais leves, repletos de cor e amor, Clarinha e Dani! Amo vocês todos e sem vocês já não há mais sentido!

% End of Acknowledgments
