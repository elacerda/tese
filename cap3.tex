%%%%%%%%%%%%%%%%%%%%%%%%%%%%%%%%%%%%%%%%%%%%%%%%%%%%%%%%%%%%%%%%%
% Tese de Doutorado / Dept. Fisica, CFM, UFSC                   %
% Lacerda@CórregoGrande - Jan/2018                              %
%%%%%%%%%%%%%%%%%%%%%%%%%%%%%%%%%%%%%%%%%%%%%%%%%%%%%%%%%%%%%%%%%

%:::::::::::::::::::::::::::::::::::::::::::::::::::::::::::::::%
%                                                               %
%                          Capítulo 3                           %
%                                                               %
%:::::::::::::::::::::::::::::::::::::::::::::::::::::::::::::::%

%***************************************************************%
%                                                               %
%                          DIG class                            %
%                                                               %
%***************************************************************%

\chapter{Classificação}
\label{sec:DIGclass}
Nosso objetivo neste trabalho é desenvolver uma maneira de caracterizar as regiões de galáxias pelo seu regime de ionização, ou seja, separar regiões SF e DIG, diferenciar componentes do DIG e ir além, servir de legado para futuros trabalhos que possam utilizar essa classificação no estudo do comportamento de diferentes propriedades estelares sob distintas componentes do ISM. Além disso, podemos analisar o víes causado pela mistura dessas diferentes componentes nas assinaturas espectrais, resolvendo o {\em conumdrum} envolvendo a espectroscopia de uma fibra (ver Seção \ref{sec:intro:partes}).

\section{O papel de $W_{H\alpha}$ na classificação das regiões: hDIG, mDIG e SFc}
\label{sec:DIGclass:WHa}

Vários trabalhos anteriores utilizam o brilho superficial de \Ha na intenção de separar regiões SF e DIG. Por exemplo \citet{Zhang.etal.2017a} argumenta que para os dados do MaNGA \citep{Bundy.etal.2015}, {\em spaxels} onde $\Sigma_{\Ha} > \Sigma_{\Ha}^{\rm SF,min} = 10^{39}$ erg$\,$s$^{-1}\,$kpc$^{-2}$ são confiavelmente dominados por SF.

Como dissemos anteriormente, nós preferimos classificar as regiões em SF e DIG baseados em $W_{\Ha}$. Vemos na classificação utilizando $\Sigma_{\Ha}$ um erro conceitual que pode ser explicado com um pequeno experimento teórico.

Imagine dois elementos de volume dominados por DIG, ambos com área superficial $A$ emitindo um fluxo $F_{\Ha}=A\times\Sigma_{\Ha}$, como na Figura \ref{fig:DIGDIG}. Assuma que o meio é opticamente translúcido para os fótons de \Ha (não há extinção), como é apropriado para regiões de DIG, de maneira que o volume inteiro seja visto. Obviamente uma operação de soma com duas regiões DIG não deve alterar a natureza da região observada. Quando na linha de visada vemos uma região ao lado da outra, medimos o mesmo brilho superficial, pois temos duas vezes o mesmo fluxo e duas vezes a mesma área, $\Sigma_{\Ha}=(2 \times F_{\Ha})/(2 \times A)$, mantendo uma classificação por um limite no brilho superfical correta. Por outro lado, quando vemos os dois elementos sobrepostos (ambos sobre a mesma linha de visada) medimos o dobro do brilho superficial, $\Sigma_{\Ha}=(2 \times F_{\Ha})/A$, fazendo com que uma operação DIG+DIG possa resultar em SF, conceitualmente errada.
%Esse viés é conceitualmente errado, pois duas regiões DIG sobrepostas deveriam continuar sendo classificadas como DIG.
Uma classificação utilizando $W_{\Ha}$ não carrega essa inconsistência por construção pois a largura equivalente final é a mesma independente da forma que os elementos são vistos.

Como veremos na Seção \ref{sec:DIGdisc:compare}, nos bojos de galáxias, onde há um percurso óptico maior, essa diferença nos critérios de classificação é de particular importância, podendo levar $\Sigma_{\Ha} > \Sigma_{\Ha}^{\rm SF,min}$ mesmo em absência de formação estelar.

Outro argumento, independente desse, é que propriedades que possuam uma dependência radial, como cor, densidade de massa estelar, quantidade de gás, entre outras, fazem com que uma classificação usando um limite fixo em não seja apropriada para todas as partes de uma galáxia. Particulamente, quando o regime de ionização do DIG é orquestrado por HOLMES




\label{sec:synvsneb}

%% End of this chapter
