%%%%%%%%%%%%%%%%%%%%%%%%%%%%%%%%%%%%%%%%%%%%%%%%%%%%%%%%%%%%%%%%%
% Tese de Doutorado / Dept. Fisica, CFM, UFSC                   %
% Lacerda@CórregoGrande - Jan/2018                              %
%%%%%%%%%%%%%%%%%%%%%%%%%%%%%%%%%%%%%%%%%%%%%%%%%%%%%%%%%%%%%%%%%

%:::::::::::::::::::::::::::::::::::::::::::::::::::::::::::::::%
%                                                               %
%                   Anexo EmLinesDataCube                       %
%                                                               %
%:::::::::::::::::::::::::::::::::::::::::::::::::::::::::::::::%


\chapter{Propriedades calculadas no \emldc}
\label{apendice:EMLprops}

\section{Extinção estimada através do decremento de Balmer}
\label{apendice:EMLprops:tauvneb}
Em um modelo que assume que entre o observador e a fonte de energia existe uma camada difusa, como uma cortina, que extingue a luz diferentemente em cada comprimento de onda, temos:
\begin{equation}
	F_\lambda^{obs} = F_\lambda^{int} e^{-\tau_\lambda}
    \label{eq:extin}
\end{equation}
\noindent onde $F_\lambda^{int}$ é o fluxo intrínseco ($F_\lambda^{obs}$, o observado) em cada comprimento de onda, $\tau_\lambda$ é a profundidade óptica para o comprimento de onda $\lambda$ (neste trabalho também o chamamos de coeficiente de extinção). O modelo de extinção de \citet{CCM1989a} nos dá uma calibração empírica da razão entre os coeficientes de extinção em um comprimento de onda e na banda V. Com isso podemos desenvolver \eqref{eq:extin} de maneira que possamos escrever uma equação para $\tau_V$:
\begin{eqnarray}
   F_\lambda^{obs} &=& F_\lambda^{int} e^{-(\frac{\tau_\lambda}{\tauV}) \tauV} \\
   q_\lambda &\equiv& \frac{\tau_\lambda}{\tauV} \\
   F_\lambda^{obs} &=& F_\lambda^{int} e^{-q_\lambda \tauV} \\
   \frac{F_\lambda^{obs}}{F_{\lambda^\prime}^{obs}} &=& \
 \frac{F_\lambda^{int} e^{-q_\lambda \tauV}}{F_{\lambda^\prime}^{int} e^{-q_{\lambda^\prime} \tauV}} \\
   \ln \left(\frac{F_\lambda^{obs}}{F_{\lambda^\prime}^{obs}}\right) &=& \
 \tauV (q_{\lambda^\prime} - q_\lambda) \ln \left(\frac{F_\lambda^{int}}{F_{\lambda^\prime}^{int}}\right) \\
   \tauV &=& \frac{1}{(q_{\lambda^\prime} - q_\lambda)} \left[\ln \
 \left(\frac{F_\lambda^{obs}}{F_{\lambda^\prime}^{obs}}\right) - \
 \ln \left(\frac{F_\lambda^{int}}{F_{\lambda^\prime}^{int}}\right)\right]
 \label{eq:tauV}
\end{eqnarray}
\noindent Nessa equação, os $q_\lambda$ são provenientes da curva de extinção adotada.

Utilizando \eqref{eq:tauV} podemos calcular qual o coeficiente de extinção para essas regiões nebulares. Para este cálculo utilizamos o fato de que a razão entre os fluxos intrínsecos das duas primeiras linhas da série de Balmer, \Ha e \Hb, varia muito um pouco com a metalicidade, a densidade e a temperatura. Usamos aqui esse valor como constante e igual a $2,86$ \citep[densidade eletrônica de $n = 100\ cm^{-3}$ e temperatura eletrônica $T_e = 10^4$ K; ][]{Osterbrock.Ferland.2006a}. Tambem utilizamos $q_{\Ha}  = 0,81775$ e $q_{\Hb} = 1,16427$ (valores da calibração empírica feita por \citeauthor{CCM1989a}). Com isso temos:
\begin{equation}
	\tauVN = 2,886\ \ln \left( \frac{F_{\Ha}^{obs}/F_{\Hb}^{obs}}{2,86} \right).
\end{equation}

O valor de $\tauVN$ varia tipicamente entre 0,6 e 0,65 (média e mediana) nas regiões aqui estudadas (ver também Seção \ref{apendice:synvsneb:tauv}).
O erro propagado para $\tauVN$ é:
\begin{eqnarray}
	\tauVN &\equiv& \tauVN(F_{\Ha}^{obs}, F_{\Hb}^{obs}) \\
	\epsilon (\tauVN) &=& \sqrt{\left(\del{\tauVN}{F_{\Ha}^{obs}}\right)^2 \
\epsilon (F_{\Ha}^{obs})^2 + \left(\del{\tauVN}{F_{\Hb}^{obs}}\right)^2 \
\epsilon (F_{\Hb}^{obs})^2 } \\
	\del{\tauVN}{F_{\Ha}^{obs}} &=& \frac{1}{F_{\Ha}^{obs} (q_{\Hb} - q_{\Ha})} \\
	\del{\tauVN}{F_{\Hb}^{obs}} &=& - \frac{1}{F_{\Hb}^{obs} (q_{\Hb} - q_{\Ha})} \\
	\epsilon (\tauVN) &=& \frac{1}{(q_{\Hb} - q_{\Ha})} \
\sqrt{\left(\frac{\epsilon (F_{\Ha}^{obs})}{F_{\Ha}^{obs}}\right)^2 + \
\left(\frac{\epsilon (F_{\Hb}^{obs})}{F_{\Hb}^{obs}}\right)^2 }
	\label{eq:etauneb}
\end{eqnarray}
\noindent Os termos de dentro da raiz em \eqref{eq:etauneb} são o inverso das relações sinal-ruído ($S/N$) de \Ha e \Hb. Como \Hb é sempre a mais ruidosa das duas, podemos aproximar a incerteza em $\tauVN$ por:
\begin{equation}
	\epsilon (\tauVN) \approx \frac{2,886}{(S/N)_{\Hb}}
	\label{eq:etaunebaprox}
\end{equation}
\noindent com valor típico $\sim 0,14$.

\section{Cálculo da taxa de formação estelar}
\label{apendice:EMLprops:SFR}
Um dos métodos mais utilizados para medida da SFR recente utiliza a linha de emissão de \Ha. Nesse método assume-se que a formação estelar é constante nos últimos $t$ anos, que deve englobar pelo menos o tempo de vida das estrelas massivas ionizantes, as quais produzem basicamente todos os fótons que geram as linhas de emissão em \Ha.

Nós queremos calibrar $\LHalpha$ como um indicador de SFR usando uma relação linear:
\begin{equation}
	\mathrm{SFR}_{\Ha} = k \times \LHalpha.
	\label{eq:SFRHa}
\end{equation}
\noindent Portanto, nosso trabalho é encontrar $k$. Faremos isso investigando a natureza dos
fótons H-ionizantes.

Chamamos $\Lambda$ o valor total de luz ($l$) que recebemos de estrelas que se formaram $t$ anos atrás. $l(t)$ pode ser qualquer função que descreve a evolução temporal de qualquer fonte radiativa genérica \emph{por unidade de massa formada}\footnote{A expressão {\em massa formada} é aqui utilizada porque quando falamos de sistemas estelares temos que diferenciar o quanto de massa se tornou em estrelas e o quanto ainda continua na forma de estrelas. Isso acontece porque a evolução estelar devolve matéria para o meio interestelar (atrvés de explosão de supernovas, nebulosas planetárias, ventos, etc). Uma SSP pode devolver para o ISM até 50\% de massa durante um tempo de Hubble, dependendo, entre outras coisas, da função de massa inicial ({\em initial mass function}; IMF).} de uma população estelar simples ({\em simple stellar population}; SSP):
\begin{equation}
	\Lambda(t) = \int_0^t l(t')\ \textrm{d}\textrm{M}(t').
	\label{eq:dLambda}
\end{equation}
\noindent Para obter $\Lambda$ nos falta saber como a massa em estrelas cresce no tempo, ou seja, saber como a SFR varia no tempo:
\begin{equation}
	\mathrm{d}\mathrm{M}(t)\ =\ \mathrm{SFR}(t)\ \mathrm{d}t
	\label{eq:dM_t}
\end{equation}

%---------------------------- Figure ----------------------------
\begin{figure}
	\centering
	\includegraphics[scale=0.62]{figuras/Nh_logt_metBase_Padova2000_salp.pdf}
	\caption[Evolução temporal do número e da taxa de fótons H-ionizantes em unidades da massa
	formada.]
	{\emph{Painel superior esquerdo}: A	evolução no tempo do número de fótons ($\mathcal{N}_H$) para 6 metalicidades (de 0,02 $Z_\odot$ até 1,58 $Z_\odot$) que compoem nossos modelos de SSP. A linha preta grossa representa a evolução utilizando metalicidade solar. \emph{Painel superior direito}: O mesmo que o \emph{painel superior esquerdo}, contudo normalizado pelo valor total de $\mathcal{N}_H$. A linha pontilhada representa 95\% do total de $\mathcal{N}_H$. Em destaque a região ao redor de 10 milhões de anos. \emph{Painel inferior}: Evolução da taxa de fótons H-ionizantes em unidades da massa formada. Também mostra o valor seguindo o código de cores para as mesmas metalicidades.}
	\label{fig:Nh_qh}
\end{figure}
%---------------------------- Figure ----------------------------

Utilizando \eqref{eq:dM_t} em \eqref{eq:dLambda} e integrando dentro do tempo do Universo ($T_U\ \sim$ 14 bilhões de anos) teremos hoje um total de luz:
\begin{eqnarray}
	\Lambda(t = T_U) &=& \int_0^{T_U} l(t)\ \textrm{d}\textrm{M}(t) \\
	&=& \int_0^{T_U} l(t)\ \mathrm{SFR}(t)\  \textrm{d}t
	\label{eq:Lambda}
\end{eqnarray}
\noindent Assumindo o caso B de recombinação do hidrogênio, um em cada 2,206 fótons ionizantes produzem um fóton de \Ha \citep{Osterbrock.Ferland.2006a}\footnote{Em um volme $\Delta V$ de uma região \hii, o número de recombinações $p + e \rightarrow {\rm H}^0$ por tempo é dado por $n_p n_e \alpha_B({\rm H}^0) \Delta V$, onde $n_e$ e $n_p$ são as densidades eletronica e protônica e $\alpha_B({\rm H}^0)$ é o coeficiente total de recombinações para o caso B. Na Tabela 2.7 de \citet{Osterbrock.Ferland.2006a} temos $\alpha_B({\rm H}^0) = 2,59\times10^{-13}$ recombinações $\times\,cm^3\,s^{-1}$ (para uma temperatura de $10\,000$ K). Durante a cascata de recombinação, alguns dos elétrons passam pela transição $n = 3 \rightarrow 2$, produzindo \Ha. Seja $\alpha_{\Ha}^{eff}({\rm H}^0)$ o coeficiente que conta apenas esse tipo de recombinação. Esse valor não é explicitamente dado no livro de \citet{Osterbrock.Ferland.2006a}, porém eles provém (na Tabela 4.7) o valor eficaz para \Hb ($\alpha_{\Hb}^{eff}({\rm H}^0) = 3,03\times10^{-14}$), juntamente com a razão entre as emissividades $j_{\Ha}/j_{\Hb} = 2,87$. Usando
$$
\frac{ \alpha_{\Ha}^{eff}({\rm H}^0) }{ \alpha_{\Hb}^{eff}({\rm H}^0) } = \frac{ j_{\Ha} / h\nu_{\Ha} }{ j_{\Hb} / h\nu_{\Hb} }
$$
\noindent (onde a razão de energias dos fótons de \Ha e \Hb aparecem porque $j$ mede energia e $\alpha$ mede um número de recombinações), nós finalmente obtemos $\alpha_{\Ha}^{eff}({\rm H}^0) = 1,17\times10^{-13} \, cm^{3}\,s^{-1}$ , ou $0,453 \times \alpha_B({\rm H}^0)$. Em outras palavras, um a cada $1/0,453 = 2,206$ recombinações produz um fóton de \Ha. Considerando a núvem em equilíbrio (o número de fotoionizações se balanceia com o de recombinações) e que a radiação $h\nu > 13,6$ eV não escape da núvem, podemos finalmente dizer que um a cada 2,206 fótons ionizantes produz um fóton \Ha.}. Esse número não varia muito em função da temperatura e da densidade nas regiões \Hii. Portanto:
\begin{equation}
	\LHalpha = h \nu_{\Ha} \frac{\mathrm{Q}_H}{2,206},
	\label{eq:LHa_recomb_theory}
\end{equation}
\noindent onde $\mathrm{Q}_H$ é a taxa de fótons H-ionizantes. Em todo este processo assume-se que nenhuma radiação ionizante escapa da nuvem e, apesar de $\LHalpha$ estar corrigido por extinção, também assume-se que a poeira não absorve muitos dos fótons com $h\nu\ > 13,6$ eV. Escrevemos $dQ_H$ como a equação \eqref{eq:dLambda}.
\begin{equation}
	Q_H(t)\ =\ \int dQ_H = \int q_H(t)\ \mathrm{d}\mathrm{M}(t)
	\label{eq:QH_t}
\end{equation}
\noindent Nas equações acima, $q_H$ (que assume o mesmo papel de $l(t)$ na Equação \ref{eq:dLambda}) é a taxa de fótons H-ionizantes por unidade de massa formada. Considerando os fótons que possam ionizar o hidrogênio ($h\nu\ \geq\ 13,6$ eV ou $\lambda\ \leq\ 912\AA$) escrevemos:
\begin{equation}
	q_H(t) = \int_0^{912\AA} \frac{l_\lambda\ \lambda}{h c} d\lambda.
	\label{eq:qH}
\end{equation}
\noindent Nesta equação, $l_\lambda$ é a luminosidade por unidade de massa formada e comprimento de onda em unidades solares $[\textrm{L}_\odot/\AA\textrm{M}_\odot]$ para uma SSP\footnote{Apesar de não escritas aqui, existem dependências com Z, IMF e isócronas em $l_\lambda$ (portanto, também em $q_H$ e todos os seus produtos)}. Com isso, nós ainda precisamos analisar como a integração de $q_H$ evolui com o tempo, para então obter a SFR. Integrando $q_H$ de hoje até $T_U$ nós obtemos o número de fótons H-ionizantes produzidos pelas fontes que emitem a luz $l$:
\begin{equation}
	\mathcal{N}_H = \int_0^{T_U} q_H(t)\ dt
	\label{eq:Nh}
\end{equation}

Utilizamos a base de populações estelares de \citet{Bruzual.Charlot.2003} com as isócronas de Padova 2000, juntamente com a IMF de \citet{Salpeter.1955a}. Neste caso, podemos ver na Figura \ref{fig:Nh_qh} a evolução temporal de $\mathcal{N}_H$ em valores absolutos (painel superior esquerdo) e relativamente ao total de $\mathcal{N}_H$ (painel superior direito). Em \citet[Figura 2b]{CidFernandes.etal.2011a} nós podemos ver a evolução temporal de $q_H$ sob todas as idades e metalicidades\footnote{Naquele estudo, o grupo \href{http://starlight.ufsc.br}{SEAGal/\STARLIGHT} utilizou as isócronas de Padova 1994 com a IMF de \citet{Chabrier.2003a}.}. A mesma figura é reproduzida no painel inferior da Figura \ref{fig:Nh_qh}. É notável que o número de fótons H-ionizantes rapidamente converge ao máximo perto de $t = 10^7$ anos. Deste mesmo gráfico mostramos que $q_H$ atinge um valor mínimo constante depois de uma certa idade. Nessa escala de tempo, dominam o regime de ionização estrelas velhas e quentes (ver Capítulo \ref{sec:DIGclass}). Também conclui-se que $\mathcal{N}_H$ é dependente da metalicidade, portanto o valor de $k$ em \eqref{eq:SFRHa} também\footnote{O valor de $k$ em nossa análise varia de 2,00 até 3,13 de acordo com a metalicidade indo de $0,02 Z_\odot$ até $1,58 Z_\odot$.}.

Utilizando \eqref{eq:dM_t} em \eqref{eq:QH_t} e com uma simplificação graças a SFR constante dentro desta escala temporal ($\mathrm{SFR}(t)\rightarrow \mathrm{SFR}$), podemos reescrever \eqref{eq:QH_t} usando \eqref{eq:Nh}:
\begin{equation}
	Q_H = \mathrm{SFR}\ \mathcal{N}_H(t_{\rm ion}\ =\ 10^7\ \textrm{anos, IMF, Z}{}_\star).
	\label{eq:QH_converge}
\end{equation}
\noindent Substituindo \eqref{eq:QH_converge} em \eqref{eq:LHa_recomb_theory} podemos escrever:
\begin{equation}
	\mathrm{SFR}_{\Ha} = \frac{2,206}{\mathcal{N}_H\ h \nu_{\Ha}} \times \LHalpha
	\label{eq:SFR_theoric}
\end{equation}
\noindent
%{\ATR [\ojo ajeita/reword essa frase!!] Este método resulta em uma SFR recente, em termos de que, baseados na Figura \ref{fig:Nh_qh}, assumimos que o tempo para produção de fótons ionizantes relevantes para estimar a SFR baseada em $\LHalpha$ é de $t_{\rm ion} = 10^7$ anos}. Finalmente, resolvendo \eqref{eq:SFR_theoric} encontramos o valor para $k$ em \eqref{eq:SFRHa}:
Quando fazemos este tipo de estudo utilizando $\LHalpha$, temos a estimativa de uma SFR recente pois assumimos que o tempo para produção de fótons ionizantes relevantes para estimar a SFR é de $t_{\rm ion} = 10^7$ anos. Porém, como vimos anteriormente através da Figura \ref{fig:Nh_qh} essa é a escala de tempo onde a disponibilidade de fótons é praticamente máxima. Finalmente, resolvendo \eqref{eq:SFR_theoric} encontramos o valor para $k$ em \eqref{eq:SFRHa}:

\begin{equation}
	\mathrm{SFR}_{\Ha}\ [\mathrm{M}_\odot\ \mathrm{yr}^{-1}] = 3,13 \times
	\left(\frac{\LHalpha}{10^8\ \mathrm{L}_\odot}\right) = 8,1 \times 10^{-42}\ \LHalpha\ [ergs\ s^{-1}]
	\label{eq:SFRNeb}
\end{equation}

Em \citet{Kennicutt.1998a} esse coeficiente é calculado utilizando diferentes modelos estelares mas utilizando a mesma IMF. Nosso valor é bem próximo daquele obtido no artigo ($7,9 \times 10^{-42}$).

% End of this chapter
