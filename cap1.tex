%%%%%%%%%%%%%%%%%%%%%%%%%%%%%%%%%%%%%%%%%%%%%%%%%%%%%%%%%%%%%%%%%
% Tese de Doutorado / Dept. Fisica, CFM, UFSC                   %
% Lacerda@Cidreira - Dez/2017                                   %
%%%%%%%%%%%%%%%%%%%%%%%%%%%%%%%%%%%%%%%%%%%%%%%%%%%%%%%%%%%%%%%%%

%:::::::::::::::::::::::::::::::::::::::::::::::::::::::::::::::%
%                                                               %
%                          Capítulo 1                           %
%                                                               %
%:::::::::::::::::::::::::::::::::::::::::::::::::::::::::::::::%

%***************************************************************%
%                                                               %
%                         Introdução                            %
%                                                               %
%***************************************************************%


\chapter{Introdução}
\label{sec:intro}

A única forma empírica de estudarmos galáxias é através da luz emitida pelos seus constituintes. Mais precisamente, das imagens e da distribuição espectral de energia (SED\footnote{{\em Spectral energy distribution} - quantidade de energia em cada comprimento de onda.}) que chegam até nossos telescópios, em terra ou no espaço. Diferentes componentes e eventos os modificam, nos possibilitando a busca de padrões e criação de modelos que se propõem a explicar sua constituição, formação e dinâmica. Atualmente, existem diversos projetos astronômicos de levantamento de informações ou mapeamento de regiões do céu, chamados de {\em surveys}, formando uma rede de gigantescos bancos de dados de imagens, espectros e metainformação. Com diferentes faixas espectrais (desde raios-$\gamma$ até micro-ondas), diferentes fontes de dados (espectros de galáxias integradas, espectroscopia de campo, imagens, monitoramento temporal de eventos) e diferentes objetivos, os {\em surveys} astronômicos permeiam por diferentes fenônmenos astrofísicos. Através dessa criação e difusão em massa de informações, nossa forma de enxergar o mundo vem se tornando cada vez mais acurada quanto ao Universo. Além de estarem formando um imenso legado de informações para futuros astrofísicos, são basilares para o desenvolvimento de novas ideias e resolução dos desafios atuais da área. Neste capítulo faço uma introdução do atual cenário com um breve resumo dos avanços que nosso grupo de astrofísica (GAS-UFSC) têm obtido através desses levantamentos, cenário no qual esta tese está inserida.


%\section{Espectros integrados versus espectroscopia de campo}
\section{O todo e as partes}
\label{sec:intro:partes}

Galáxias são formadas por uma complexa mistura de gás, poeira, estrelas e matéria escura, distribuídas em discos, bulbos e halos. Os primeiros grandes levantamentos de dados espectrais (SDSS\footnote{\em Sloan Digital Sky Survey}, \citealt{York.etal.2000a}; COSMOS\footnote{\em Cosmic Evolution Survey}, \citealt{Scoville.etal.2007}; ALHAMBRA\footnote{\em Advanced Large Homogeneous Area Medium-Band Redshift Astronomical survey}, \citealt{Moles.etal.2008}; são alguns exemplos) tratavam galáxias como uma fonte puntual de energia, conhecido com espectroscopia de uma fibra ({\em single-fiber spectroscopy}\footnote{Um espectro formado apenas de uma determinada área de abertura observada.}). Apesar dessa limitação, muito se aprendeu (e ainda se aprende) sobre a formação e evolução das galáxias. Exemplos incluem a conexão entre o poder do AGN\footnote{\em Active galactic nucleus} e as populações estelares \citep{Kauffmann.etal.2003a}; a relação entre a taxa de formação estelar (SFR\footnote{\em Star-formation rate}) e a massa estelar das galáxias \citep{Brinchmann.etal.2004a}; a relação massa-metalicidade (MZR\footnote{\em Mass-metallicity relation}; \citealt{Tremonti.etal.2004a}); a evolução química e a história de formação estelar das galáxias\citep{CidFernandes.etal.2007, Asari.etal.2007a}; relação massa estelar-metalicidade \citep{ValeAsari.etal.2009a}; e mais importantes para o escopo desta tese, a revelação de uma imensa e esquecida população de galáxias aposentadas ionizadas por HOLMES\footnote{{\em Hot low-mass evolved stars}, estrelas quentes de baixa massa em alto estado de evolução.} \citep{Stasinska.etal.2008a, CidFernandes.etal.2010a, CidFernandes.etal.2011a}. Entretanto, esse é um dos mais significantes problemas desse tipo de {\em survey}.

Podemos perceber que qualquer propriedade que varie em função da posição dentro da galáxia será erroneamente estimada quando há apenas um espectro representando a galáxia inteira. Quando estimamos propriedades referentes a diferentes regimes de ionização na galáxia, como a metalicidade nebular por exemplo, também temos problemas. Nesse caso, devemos levar em conta apenas os fótons gerados nas regiões de formação estelar (SF\footnote{{\em Star-forming}}), isolando-os daqueles que vêm de outros regimes nebulares, como o gás ionizado difuso (DIG\footnote{{\em Diffuse ionized gas}}), fotoionização pelo núcleo ativo ou estrelas velhas. Dessa forma, para um estudo mais preciso das propriedades derivadas dos espectros integrados e, por consequência, do viés causado por construção, um melhor entendimendo desses efeitos se faz necessário.

Um grande passo nessa direção foi dado com a criação dos {\em surveys} de espectroscopia de campo (IFS\footnote{{\em Integral field spectroscopy}}). Através da IFS podemos desvencilhar essa mistura de partes distintas, pois nessa técnica de observação temos espectros para cada parte da galáxia. Assim, para cada par espacial ($x,y$) temos uma dimensão espectral $\lambda$. Quanto menores sejam os píxels cobrindo uma mesma área (melhor resolução espacial) e melhor resolução espectral cobrindo um maior intervalo de comprimento de onda teremos uma melhor definição da localização e da assinatura espectral de cada uma das partes. Diversos {\em surveys} IFS já estão finalizados e com seus dados disponíveis publicamente (CALIFA\footnote{\em Calar Alto Legacy Integral Field Area survey} DR3\footnote{\em Data-release 3}, \citealt{SFSanchez.DR3.2016}; PINGs\footnote{\em PPAK IFS Nearby Galaxies survey}, \citealt{RosalesOrtega.etal.2010}), outros ainda estão em fase de observação e com alguns dados já disponíveis (MaNGA\footnote{\em Mapping nearby Galaxies at Apache Point Observatory} SDSS-IV DR13, \citealt{MaNGADR1.2017}; SAMI\footnote{equipamento e {\em survey} são homônimos - {\em Sydney-AAO Multi-object Integral-field spectrograph}} DR1, \citealt{SAMIDR1.2017}). Com o desenvolvimento de novos equipamentos como o MUSE\footnote{{\em The Multi Unit Spectroscopic Explorer} - \href{https://www.eso.org/sci/facilities/develop/instruments/muse.html}{https://www.eso.org/sci/facilities/develop/instruments/muse.html}} e o SITELLE\footnote{{\em Spectromètre Imageur à Transformée de Fourier pour l'Etude en Long et en Large de raies d'Emission} - \href{http://cfht.hawaii.edu/Instruments/Sitelle/}{http://cfht.hawaii.edu/Instruments/Sitelle/}} poderemos estudar galáxias e suas interações com ainda mais detalhes.

Nessa direção, este trabalho usa os dados de IFS do \CALS para estudar a importância e a caracterização do DIG em diferentes regiões de galáxias de todos os tipos morfológicos, que resultou no artigo no Apêndice \ref{apendice:DIGpaper0} \citep{Lacerda.etal.2018}. A completa cobertura de galáxias com diferentes morfologias e diferentes inclinações faz do CALIFA um {\survey} ideial para esse tipo de estudo, mesmo sabendo que a resolução espacial não nos permite uma descrição detalhada das diferentes componentes do meio interestelar (ISM\footnote{\em Interstellar medium}). IFS com melhor resolução já existem \citep{Sanchez.etal.2015MUSE, Vogt.etal.2017a, RousseauNepton.etal.2017}, mas como cobrem tão poucos objetos não podemos usá-los para um estudo mais geral.


\section{Gás ionizado difuso (DIG)}
\label{sec:intro:DIG}

\subsection{Primeiras detecções}
O DIG foi detectado pela primeira vez no disco Galactico através de linhas de emissão fracas fora de regiões \Hii\footnote{Regiões formadoras de estrelas; são formadas por imensas nuvens de gás molecular, originado pelo esfriamento de gás do meio interestelar, que se fragmentam formando estruturas menores e cada vez mais densas.} clássicas \citep{Reynolds.PhD.1971}. Observações de galáxias espirais de lado através de imageamento em \Ha \citep{Dettmar.1990, HoopesWaltGreen.1996, HoopesWaltRand.1999} mostraram a existência de DIG a grandes distâncias do plano galáctico. \cite{Oey.etal.2007}, estudando 109 galáxias do SINGS\footnote{\em Spitzer Infrared Nearby Galaxies Survey}, chegaram a conclusão que emissão difusa em \Ha está presente em galáxias de todos os tipos morfológicos e representa $\sim60\%$ da emissão total em \Ha, independentemente do tipo morfológico ou da SFR total.

\subsection{Fonte de ionização do DIG}
Fótons de estrelas massivas do tipo OB escapando das regiões \Hii é a fonte de ionização mais comumente adotada para explicar o DIG (veja o review em \citealt{Haffner.etal.2009}). Entretanto, razões de linhas como \nii/\Ha, \sii/\Ha, and \oiii/\Hb crescem com a altura em relação ao plano galáctico, fazendo com que seja necessário a inclusão de fontes adicionais (ou alternativas) de ionização.


\section{O GAS-UFSC e o IAA-CSIC}
\label{sec:intro:UFSCeIAA}

Nos últimos anos nosso grupo de Astrofísica (GAS-UFSC) aqui na Universidade Federal de Santa Catarina vem trabalhando com dados de diversos {\em surveys}. Nosso grupo foi pioneiro no estudo das propriedades físicas das populações estelares de aproximadamente um milhão de galáxias do \SDSS através do projeto SEAGal/\starlight\footnote{\href{http://starlight.ufsc.br}{http://starlight.ufsc.br}} publicando diversos artigos importantes e amplamente citados \citep[e.g., ][]{CidFernandes.etal.2005a, Mateus.etal.2006a, Stasinska.etal.2006a, Asari.etal.2007a, Stasinska.etal.2008a, CidFernandes.etal.2011a}.

Desde antes e durante esse trabalho, participamos de um projeto entre nosso grupo de populações estelares aqui no GAS com pesquisadores do Instituto de Astrofísica de Andalucía (IAA), na cidade de Granada, Comunidade autônoma de Andalucía, ao sul da Espanha. Esse instituto pertence ao {\em Consejo Superior de Investigaciones Científicas} (CSIC), o maior órgão público (estatal) de pesquisas científicas na Espanha, e o terceiro maior da Europa. Conta com pesquisadores participantes do \CALS, funcionando como centro físico do projeto. A pesquisadora Rosa M. González Delgado, coorientadora deste trabalho, uma das principais líderes do projeto e que também atuou como Pesquisadora Visitante Especial (PVE-CsF) aqui na UFSC; Rubén García Benito, que faz parte do grupo de redução dos dados do {\em survey}; e Enrique Pérez, do grupo de populações estelares, já trabalham em nossa parceria e possuem conhecimento e domínio das técnicas exploradas por nosso projeto, além de participarem ativamente do desenvolvimento do CALIFA. Durante os últimos cinco anos nosso grupo de populações estelares no CALIFA publicou diversos artigos \citep[e.g.,][]{Perez.etal.2013a, GonzalezDelgado.etal.2014a, GonzalezDelgado.etal.2014b, GonzalezDelgado.etal.2015a, GonzalezDelgado.etal.2016a, deAmorim.etal.2017, GonzalezDelgado.etal.2017, RGB.etal.2017, Lacerda.etal.2017}. Paralelamente participamos de diversos congressos e conferências publicando nossos resultados. Detalhes técnicos e comparações entre {\em surveys} IFS podem ser encontrados em \citet{Andre2015}.


\subsection{Artigo - CALIFA, the Calar Alto Legacy Field Area survey IV. Third public data release.}
\label{sec:intro:UFSCeIAA:DR3}

Houveram três lançamentos públicos dos dados armazenados pelo CALIFA durante o decorrer do projeto, finalizando com o DR3\footnote{\em Data-release 3} (\citealt{SFSanchez.DR3.2016}, que também pode ser encontrado no Apêndice \ref{apendice:SFSanchezDR3} deste trabalho). Este DR conta com espectros de 667 galáxias ($\sim 1,5$ milhões de espectros) com tipos morfológicos cobrindo toda a classificação de Hubble e redshifts variando entre 0.005 e 0.03 (distâncias de 20 a 130 Mpc).

Nosso grupo de populações estelares se encarregou de escrever alguns programas de análise e gerar as imagens nas quais nos embasamos para avaliar a qualidade da síntese de populações estelares com o \starlight para distintas posições em cada galáxia, além de realizarmos uma comparação com os resultados para a amostra do DR1 \citep{Husemann.etal.2013a}.  e do DR2 \citealt{GarciaBenito.etal.2015a}. Durante esse trabalho notamos que os resíduos reduziram sensivelmente desde a última versão. Esta mesma análise nos ajudou a melhorar a máscara de remoção de linhas telúricas\footnote{Linhas provenientes de fenômenos que ocorrem na Terra.} dos espectros. Também verificamos que os erros relacionados aos espectros observados possuem uma distribuição muito próxima a uma gaussiana.


\section{Este trabalho}
\label{sec:intro:estetrabalho}

Essa tese é um apanhado de alguns dos trabalhos no qual participei durante o tempo do doutorado, com foco principal no artigo sobre a natureza das linhas de emissão das regiões das galáxias do CALIFA, separando aquelas SF daquelas melhores caracterizadas como DIG. No Cap.\ \ref{sec:amostra} apresento a amostra de galáxias utilizadas neste trabalho. O artigo principal será dividido em dois capítulos. A apresentação do método de caracterização está no Cap.\  \ref{sec:hDIGmDIGSF}, e no próximo (Cap.\ \ref{sec:discussion}) fazemos a discussão sobre esse método de classificação de regiões e a análise das regiões classificadas. Por fim, demais artigos e trabalhos são apresentados no Cap.\ \ref{sec:demaisartigos}.

% End of this chapter
