%%%%%%%%%%%%%%%%%%%%%%%%%%%%%%%%%%%%%%%%%%%%%%%%%%%%%%%%%%%%%%%%%
% Tese de Doutorado / Dept. Fisica, CFM, UFSC                   %
% Lacerda@Cidreira - Dez/2017                                   %
%%%%%%%%%%%%%%%%%%%%%%%%%%%%%%%%%%%%%%%%%%%%%%%%%%%%%%%%%%%%%%%%%

%:::::::::::::::::::::::::::::::::::::::::::::::::::::::::::::::%
%                                                               %
%                          Capítulo 1                           %
%                                                               %
%:::::::::::::::::::::::::::::::::::::::::::::::::::::::::::::::%

%***************************************************************%
%                                                               %
%                         Introdução                            %
%                                                               %
%***************************************************************%


\chapter{Introdução}
\label{sec:intro}

A única forma empírica de estudarmos galáxias é através da luz emitida pelos seus constituintes. Mais precisamente, das imagens e da distribuição espectral de energia (SED\footnote{{\em Spectral energy distribution} - quantidade de energia em cada comprimento de onda.}) que chegam até nossos telescópios, em terra ou no espaço. Diferentes componentes e eventos os modificam, nos possibilitando a busca de padrões e criação de modelos que se propõem a explicar sua constituição, formação e dinâmica. Atualmente, existem diversos projetos astronômicos de levantamento de informações ou mapeamento de regiões do céu, chamados de {\em surveys}, formando uma rede de gigantescos bancos de dados de imagens, espectros e metainformação. Com diferentes faixas espectrais (desde raios-$\gamma$ até micro-ondas), diferentes fontes de dados (espectros de galáxias integradas, espectroscopia de campo, imagens, monitoramento temporal de eventos) e diferentes objetivos, os {\em surveys} astronômicos permeiam por diferentes fenônmenos astrofísicos. Através dessa criação e difusão em massa de informações, nossa forma de enxergar o mundo vem se tornando cada vez mais acurada quanto ao Universo. Além de estarem formando um imenso legado de informações para futuros astrofísicos, são basilares para o desenvolvimento de novas ideias e resolução dos desafios atuais da área. Neste capítulo faço uma introdução do atual cenário com um breve resumo dos avanços que nosso grupo de astrofísica (GAS-UFSC) têm obtido através desses levantamentos e também do assunto no qual essa tese se baseia.


%\section{Espectros integrados versus espectroscopia de campo}
\section{O todo e as partes}
\label{sec:intro:partes}

Galáxias são formadas por uma complexa mistura de gás, poeira, estrelas e matéria escura, distribuídas em discos, bulbos e halos. Os primeiros grandes levantamentos de dados espectrais (SDSS, \citealt{York.etal.2000a}; COSMOS, \citealt{Scoville.etal.2007}; ALHAMBRA, \citealt{Moles.etal.2008}; são alguns exemplos) tratavam galáxias como uma fonte puntual de energia, conhecido com espectroscopia de uma fibra ({\em single-fiber spectroscopy}\footnote{Um espectro formado apenas de uma determinada área de abertura observada.}. Apesar dessa limitação, muito se aprendeu (e ainda se aprende) sobre a formação e evolução das galáxias. Entretanto, esse é um dos mais significantes problemas desse tipo de {\em survey}. Ao estimar a metalicidade nebular, por exemplo, devemos levar em conta apenas os fótons gerados nas regiões de formação estelar (SF\footnote{{\em Star-forming}}), isolando-os daqueles que vêm de outros regimes nebulares, como o gás ionizado difuso (DIG\footnote{{\em Diffuse ionized gas}}), fotoionização pelo núcleo ativo ou estrelas velhas. Também podemos perceber que qualquer propriedade que varie em função da posição dentro da galáxia será erroneamente estimada quando temos apenas um espectro representando a galáxia inteira. Para um estudo mais preciso das propriedades derivadas dos espectros integrados e, por consequência, do viés causado por sua natureza é necessário um melhor entendimendo desses efeitos.

Um grande passo nessa direção foi dado com a criação dos {\em surveys} de espectroscopia de campo (IFS\footnote{{\em Integral field spectroscopy}}). Através da IFS podemos desvencilhar essa mistura de partes distintas, já que nessa técnica de observação temos espectros para cada parte da galáxia. Dessa forma, para cada par espacial ($x,y$) temos uma dimensão espectral $\lambda$. Quanto mais precisos sejam $x$, $y$ (resolução espacial) e $\lambda$ (resolução espectral) teremos uma melhor definição da localização e da assinatura espectral de cada uma das partes. Diversos {\em surveys} IFS já estão finalizados e com seus dados disponíveis publicamente (CALIFA, \citealt{Sanchez.etal.2012a}; PINGs, \citealt{RosalesOrtega.etal.2010}), outros ainda estão em fase de observação e com alguns dados já disponíveis (MaNGA SDSS-IV DR13, \citealt{MaNGADR1.2017}; SAMI DR1, \citealt{SAMIDR1.2017}). Com o desenvolvimento de novos equipamentos como o MUSE\footnote{{\em The Multi Unit Spectroscopic Explorer} - \url{https://www.eso.org/sci/facilities/develop/instruments/muse.html}}) poderemos estudar galáxias e suas interações com ainda mais detalhes.


\section{O GAS-UFSC e o IAA-CSIC}
\label{sec:intro:UFSCeIAA}

Nos últimos anos nosso grupo de Astrofísica (GAS-UFSC) aqui na Universidade Federal de Santa Catarina vem trabalhando com dados de diversos {\em surveys}. Nosso grupo foi pioneiro no estudo das propriedades físicas das populações estelares de aproximadamente um milhão de galáxias do \SDSS através do projeto SEAGal/\starlight\footnote{\href{http://starlight.ufsc.br}{http://starlight.ufsc.br}} publicando diversos artigos importantes e amplamente citados \citep[e.g., ][]{CidFernandes.etal.2005a, Mateus.etal.2006a, Stasinska.etal.2006a, Asari.etal.2007a, Stasinska.etal.2008a, CidFernandes.etal.2011a}.

Desde antes e durante esse trabalho, participamos de um projeto entre nosso grupo de populações estelares aqui no GAS com pesquisadores do Instituto de Astrofísica de Andalucía (IAA), na cidade de Granada, Comunidade autônoma de Andalucía, ao sul da Espanha. Esse instituto pertence ao {\em Consejo Superior de Investigaciones Científicas} (CSIC), o maior órgão público (estatal) de pesquisas científicas na Espanha, e o terceiro maior da Europa. Conta com pesquisadores participantes do \CALS, funcionando como centro físico do projeto. A pesquisadora Rosa M. González Delgado, coorientadora deste trabalho, uma das principais líderes do projeto e que também atuou como Pesquisadora Visitante Especial (PVE-CsF) aqui na UFSC; Rubén García Benito, que faz parte do grupo de redução dos dados do {\em survey}; e Enrique Pérez, do grupo de populações estelares, já trabalham em nossa parceria e possuem conhecimento e domínio das técnicas exploradas por nosso projeto, além de participarem ativamente do desenvolvimento do CALIFA. Durante os últimos cinco anos nosso grupo de populações estelares no CALIFA publicou diversos artigos \citep[e.g.,][]{Perez.etal.2013a, GonzalezDelgado.etal.2014a, GonzalezDelgado.etal.2014b, GonzalezDelgado.etal.2015a, GonzalezDelgado.etal.2016a, deAmorim.etal.2017, GonzalezDelgado.etal.2017, RGB.etal.2017, Lacerda.etal.2017}. Paralelamente participamos de diversos congressos e conferências publicando nossos resultados. Detalhes técnicos e comparações entre {\em surveys} IFS podem ser encontrados em \citet{Andre2015}.


\subsection{Artigo - CALIFA, the Calar Alto Legacy Field Area survey IV. Third public data release.}
\label{sec:intro:UFSCeIAA:DR3}

Houveram três lançamentos públicos dos dados armazenados pelo CALIFA durante o decorrer do projeto, finalizando com o DR3\footnote{\em Data-release 3} (\citealt{SFSanchez.DR3.2016}, que também pode ser encontrado no Apêndice \ref{apendice:SFSanchezDR3} deste trabalho). Este DR conta com espectros de 667 galáxias ($\sim 1,5$ milhões de espectros) com tipos morfológicos cobrindo toda a classificação de Hubble e redshifts variando entre 0.005 e 0.03 (distâncias de 20 a 130 Mpc).

Nosso grupo de populações estelares se encarregou de escrever alguns programas de análise e gerar as imagens nas quais nos embasamos para avaliar a qualidade da síntese de populações estelares com o \starlight para distintas posições em cada galáxia, além de realizarmos uma comparação com os resultados para a amostra do DR1 \citep{Husemann.etal.2013a}.  e do DR2 \citealt{GarciaBenito.etal.2015a}. Durante esse trabalho notamos que os resíduos reduziram sensivelmente desde a última versão. Esta mesma análise nos ajudou a melhorar a máscara de remoção de linhas telúricas\footnote{Linhas provenientes de fenômenos que ocorrem na Terra.} dos espectros. Também verificamos que os erros relacionados aos espectros observados possuem uma distribuição muito próxima a uma gaussiana.


\section{Este trabalho}
\label{sec:intro:estetrabalho}

Essa tese é um apanhado de alguns dos trabalhos no qual participei durante o tempo do doutorado, com foco principal no artigo sobre a natureza das linhas de emissão das regiões das galáxias do CALIFA, separando aquelas SF daquelas melhores caracterizadas como DIG. No Cap.\ \ref{sec:amostra} apresento a amostra de galáxias utilizadas neste trabalho. O artigo principal será dividido em dois capítulos. A apresentação do método de caracterização está no Cap.\  \ref{sec:hDIGmDIGSF}, e no próximo (Cap.\ \ref{sec:discussion}) fazemos a discussão sobre esse método de classificação de regiões e a análise das regiões classificadas. Por fim, demais artigos e trabalhos são apresentados no Cap.\ \ref{sec:demaisartigos}.

% End of this chapter
