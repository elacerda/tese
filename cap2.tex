%%%%%%%%%%%%%%%%%%%%%%%%%%%%%%%%%%%%%%%%%%%%%%%%%%%%%%%%%%%%%%%%%
% Tese de Doutorado / Dept. Fisica, CFM, UFSC                   %
% Lacerda@CórregoGrande - Jan/2018                              %
%%%%%%%%%%%%%%%%%%%%%%%%%%%%%%%%%%%%%%%%%%%%%%%%%%%%%%%%%%%%%%%%%

%:::::::::::::::::::::::::::::::::::::::::::::::::::::::::::::::%
%                                                               %
%                          Capítulo 2                           %
%                                                               %
%:::::::::::::::::::::::::::::::::::::::::::::::::::::::::::::::%

%***************************************************************%
%                                                               %
%                           Amostra                             %
%                                                               %
%***************************************************************%

\chapter{Amostra}
\label{sec:sample}

Houveram dois lançamentos públicos de dados do CALIFA (DR2, \citealt{GarciaBenito.etal.2015a}, Apêndice \ref{apendice:DR2}; DR3, \citealt{SFSanchez.DR3.2016}) ao longo deste trabalho. Este último, o {\em data-release} final do {\em survey}, descreve uma amostra de 667 galáxias ($\sim 1,5$ milhões de espectros) com tipos morfológicos cobrindo toda a classificação de Hubble e redshifts variando entre 0.005 e 0.03 (distâncias de 20 a 130 Mpc). Neste capítulo vamos descrever nossa amostra de regiões de galáxias, todas presentes no DR3 do CALIFA, e demais processos e produtos envolvidos neste estudo.
%, entre eles, talvez o mais importante para nossa finalidade, o ajuste das linhas de emissão presentes nos espectros residuais.
% descrever a fonte empírica do nosso trabalho, os espectros observados
% todas as características da amostra utilizada para nosso estudo, que faz parte do DR3 do CALIFA.

\section{Definição da amostra deste trabalho}
\label{sec:sample:definicao}

A amostra de galáxias deste trabalho faz parte do DR3 CALIFA. Escolhemos 391 galáxias com dados disponíveis no formato COMBO, formado pela união entre as duas configurações de observação do CALIFA, obtendo espectros que cobrem de 3650--6850 \AA. A resolução espectral é $\sim 6$ \AA\ em largura à meia altura ({\em full width at half maximum}; FWHM) com um campo de observação ({\em field-of-view}; FoV) um pouco maior que 1 arcmin${}^2$, e {\em spaxels} com área de $1 \times 1$ arcsec$^2$, porém a resolução espacial é de cerca de $\sim 2.5$ arcsec. Isso corresponde a 0.2--1.5 kpc (0.8 kpc na mediana) no intervalo de distâncias que se encontram nossos objetos (20--123 Mpc). \edu{Fig ?}.

Essas galáxias se distribuem morfologicamente como segue: 57 elípticas, 47 S0--S0a, 62 Sa--Sab, 67 Sb, 70 Sbc e 88 Sc ou mais tardias. Essas classes morfológicas serão utilizadas neste trabalho para avaliar como as componentes de nossa classificação (hDIG, mDIG e SF) variam através da sequência de Hubble. Sistemas irregulares (morfológicamente distorcidos como aqueles estudados por \citealt{Wild.etal.2014, BB.etal.2015b, BB.etal.2015a, CortijoFerrero.etal.2017a, CortijoFerrero.etal.2017b}) foram removidos de nossa análise. Além da diversidade de tipos morfológicos, a amostra também abarca objetos com inclinações desde {\em edge-on} até {\em face-on}.

Como comentamos na Seção \ref{sec:intro:UFSCeIAA:SLCAL}, todos os cubos de dados foram pré-processados através do \pycasso, descrito em \citet{CidFernandes.etal.2013a} e \citet{deAmorim.etal.2017}. De forma resumida, após mascararmos dados espúrios e regiões de muito baixo sinal-ruído ({\em signal-to-noise}; SN), os spaxels são reamostrados em zonas de {\em Voronoi}. Nesse tipo de agrupamento espacial, espectros de regiões vizinhas são somados de forma que obtenhamos SN 20 em uma janela de 90 \AA\ no contínuo ao redor de 5635 \AA. Nossa amostra engloba $307\,958$ zonas ($\sim$ 800 por galáxia). Os espectros das zonas são processado pelo código \starlight \citep{CidFernandes.etal.2005a} modelando um espectro $M_\lambda$ para o contínuo estelar de cada zona. Trabalhos anteriores utilizaram essa {\em pipeline} para estudos das populações estelares \citep{Perez.etal.2013, GonzalezDelgado.etal.2014b, GonzalezDelgado.etal.2015a, GonzalezDelgado.etal.2016a, GonzalezDelgado.etal.2017}. O artigo base desta tese utiliza os espectros residuais para estudar propriedades de linhas de emissão.

Nem todas as regiões das galáxias que estudamos têm medidas das linhas espectrais necessárias para este estudo. Essa falta não é um defeito de observação e sim uma característica intrínseca de determinada região. As linhas em emissão geralmente estão ligadas ao gás e nem todas as partes da galáxia possuem gás. O processo de ajuste das linhas de emissão foi feito utilizando o {\sc sherpa} IFU line fitting software \citep[SHIFU;][]{RGB.etal.2017}, baseado no pacote {\sc ciao's sherpa} \citep{Freeman.etal.2001, Doe.etal.2007}. Esse programa ajusta perfis gaussianos nas linhas de emissão presentes nos espectros residuais, além de estimar os erros envolvidos neste processo. Um exemplo pode ser observado na Fig.\ \ref{fig:rgbline}. Nela vemos a linha de \Hb na zona central do objeto UGC00148 e o ajuste feito pelo programa.

%---------------------------- Figure ----------------------------
\begin{figure}
	\centering
	\includegraphics[scale=0.4]{figuras/K0012-zone0-Hb.pdf}
	\caption[Exemplo de ajuste de linha de emissão]
	{Espectro na região da linha de \Hbeta em emissão para a zona central da galáxia UGC00148 (objeto
CALIFA 12) juntamente com o melhor ajuste utilizando uma gaussiana. Em destaque a amplitude (A), o
comprimento de onda central (x) e o desvio padrão neste ajuste (\sigma).}
	\label{fig:rgbline}
\end{figure}
%---------------------------- Figure ----------------------------

Essas medidas podem ser muito sensíveis no caso das linhas serem fracas, como é o caso de \Hb. Nosso estudo usa basicamente o fluxo de \Ha, que é muito menos afetado por incertezas. De fato, a mediana de SN$_{\Ha}$ para todas as zonas é 16 e apenas em 5\% dos casos SN$_{\Ha} < 1$. No Apêndice \ref{apendice:organeb} apresentamos o programa criado para organizar os resultados dos ajustes das linhas de emissão bem como alguns exemplos de utilização.

%---------------------------- Figure ----------------------------
\begin{figure}
\includegraphics[scale=0.9]{figuras/fig_maps_class_faceon_paper.pdf}
 \caption[Imagem \SDSS e mapas de $\Sigma_{{\rm H}\alpha}$ e $W_{{\rm H}\alpha}$]
 {Imagens do \SDSS e os mapas de $\Sigma_{\Ha}$ e $W_{\Ha}$ para algumas galáxias do CALIFA. Os mapas da coluna mais à direita mostram $W_{\Ha}$ com as cores saturadas em 3 e 14 \AA, evidenciando a classificação proposta para hDIG, mDIG e SFc. Anéis elípticos tracejados demarcam distâncias radiais ao núcleo de $R = 0.5$, 1.0, 1.5, \dots em unidades do raio de meia luz ({\em half-light radius}; HLR). Píxels em branco são fontes externas, como estrelas de campo, ou outros artefatos.}
 \label{fig:ExampleMaps}
\end{figure}
%---------------------------- Figure ----------------------------

Uma seleção de galáxias de nossa amostra com imagem retirada do \SDSS, assim como os mapas de $\Sigma_{\Ha}$ e $W_{\Ha}$ pode ser visto na Figura \ref{fig:ExampleMaps}. Elipses tracejadas marcam até 3 HLR, em passos de 0.5 HLR. Como demonstrado em \citet{Perez.etal.2013}, \citet{Sanchez.etal.2014} e \citet{GonzalezDelgado.etal.2016a}, o HLR é uma boa unidade para comparação entre galáxias de diferentes tamanhos. Para nossa amostra, o HLR = $3.9 \pm 1.7$ kpc (média $\pm$ dispersão). Em galáxias espirais, podemos associar $R > 1$ HLR com o disco e $R < 0.5$ com o bojo. Para sistemas muito inclinados $R$ perde o sentido, porém essa é uma limitação que não afeta nossos resultados.

Nos mapas presentes na Figura \ref{fig:ExampleMaps} podemos notar algumas regiões mais largas. Essas regiões correspondem às zonas de {\em Voronoi}, usadas para garantir a qualidade dos espectros para serem processados pelo \starlight. Como podemos ver essas regiões são proeminentes nas partes externas, menos brilhantes, das galáxias. Porém, até 1 HLR, 97\% dos spaxels possuem SN > 20, portanto nenhuma binagem espacial é feita. Das $309\,958$ zonas em nossa amostra, $274\,534$ (89\%) são formadas de apenas 1 spaxel. As zonas restantes são formadas de 6 spaxels na mediana.

Parte de nossa análise que segue nos próximos capítulos foi baseada na estatística de $W_{\Ha}$ dos espectros de nossa amostra. É certo que o tamanho diferente das zonas introduz alguma distorção em nossos resultados. Seus efeitos serão discutidos mais adiante, porém podemos antecipar que não afetam os resultados principais apresentados nesta tese.


% \subsection{Mascarando elementos e removendo {\em outliers}}
% \label{sec:sample:mask}
%
% Para que possamos focar nossos estudos nas regiões de formação estelar, aplicamos uma máscara nos
% dados selecionando as regiões que possuam:
% \begin{itemize}
%   \setlength\itemsep{0.2cm}
%   \item medidas do fluxo integrado das linhas de \Hbeta, \oIII, \Halpha e \nII com relação
% sinal-ruído maior do que 3;
%   \item medidas para as seis propriedades comparadas neste capítulo:
%   \begin{itemize}
%     \item coeficiente de extinção proveniente da síntese - $\tauVS$;
%     \item coeficiente de extinção estimado através do decremento de Balmer - $\tauVN$;
%     \item densidade superficial da taxa de formação estelar calculado através da síntese -
% $\SigmaSFR$;
% 	\item densidade superficial da taxa de formação estelar calculado através da luminosidade de
% \Halpha - $\SigmaSFRN$;
% 	\item metalicidade média das populações estelares, pesada pela massa estelar - $\meanM{\log
% Z_\star}$;
% 	\item metalicidade nebular - $\log(O/H)$.
%   \end{itemize}
%   \item fração de luz proveniente de populações estelares jovens maior que 0.05 (5\%) ($x_Y >
% 0.05$);
%   \item $\tauV$ e $\tauVN$ maiores que 0.05;
%   \item mais do que cinco zonas contribuindo para o cálculo dos perfis radiais;
%   \item distância ao núcleo maior que 70\% do raio que contém metade da luz ({\em half-light
%  radius} - HLR) e menor que 3 HLR.
% \end{itemize}
% \noindent O que aqui chamamos de população jovem discutiremos um pouco mais adiante, na Sec.
% \ref{sec:synvsneb:SFR}. A última imposição é feita para que não haja contaminação por zonas
% do bojo da galáxia (partes centrais onde as linhas são produzidas por diferentes fenômenos físicos,
% relacionados a um núcleo ativo). Esse valor (0.7 HLR) foi definido por nossos colaboradores
% analisando as curvas de brilho das galáxias e representa um valor máximo para localização de zonas
% centrais.
%
% Na Fig.\ \ref{fig:histosample} podemos ver os histogramas normalizados (a integral dentro do
% intervalo do histograma é 1) de algumas propriedades de modo que evidencie os efeitos da máscara
% que forma nossa amostra. Em vermelho temos as 226176 regiões em 305 galáxias e, em azul, as 16479
% zonas de 184 galáxias (19 Sa, 38 Sb, 59 Sbc, 55 Sc e 13 Sd). É notável que nossa seleção busca zonas
% mais densas e mais jovens (maior fração de populações jovens diminuindo a idade média). O corte mais
% brusco em nossa amostra é devido a baixa relação sinal-ruído da linha de \oIII ($S/N_{\oIII} < 3$)
% em 91142 zonas.
% \begin{figure}
% 	\centering
% 	\includegraphics[width=0.99\textwidth]{figuras/histosample.pdf}
% 	\caption[Histogramas: densidade superficial de massa, idade média, fração de populações jovens e
% 	relação axial.]
% 	{Histogramas da densidade superficial de massa ({\em painel a}), idade média das populações
% estelares ({\em painel b}), fração em luz proveniente de populações jovens ($x_Y \equiv x_Y(t_\star <
% 31.62$ milhões de anos, {\em painel c}) e relação axial ({\em painel d}). Em vermelho temos a
% distribuição de valores de 226176 regiões em 305 galáxias e, em azul, a de 16479 zonas de 184
% galáxias resultantes da seleção. Em cada gráfico temos os valores da média, mediana, desvio padrão,
% máximo e mínimo de cada distribuição.}
% 	\label{fig:histosample}
% \end{figure}
%
% \subsection{Classificação Morfológica}
% \label{sec:sample:morf}
%
% Com tipos morfológicos variando entre Sa e Sd, massas estelares entre $10^9$ e $10^{11.5}\ M_\odot$
% e populações estelares com idades médias entre $10^8$ e $10^{10}$ anos, podemos ver na Fig.
% \ref{fig:amostraMorf} que as galáxias se ordenam de forma interessante quando agrupadas por tipo
% morfológico, anticorrelacionando com a idade média estelar e a massa estelar ($M_\star$ e $t_\star$)
% e correlacionando com a fração de luz proveniente das população jovens ($x_Y \equiv x_Y(t_\star <
% 31.62$ milhões de anos)). Cada galáxia contribui com um ponto em cada painel deste gráfico, ou seja,
% são propriedades integradas. Os intervalos entre primeiro e terceiro quartil quase não se sobrepõem
% quando analisamos as classes morfológicas por idade média.
%
% Esse resultado parece ser interessante visto que a classificação morfológica foi feita por
% colaboradores do CALIFA totalmente através de inspeção visual das imagens na banda-r do \SDSS das
% mesmas galáxias. Vemos também que as galáxias tipo Sd possuem as populações estelares mais jovens e
% menos massivas na média. Por ser um fenômeno apenas de posição do referencial de observação não
% deveríamos ver preferência por valor de relação axial ($b/a$) quando dividimos em classes
% morfológicas, o que realmente acontece.
%
% \begin{figure}
% 	\centering
% 	\includegraphics[width=0.99\textwidth]{figuras/sample_realsample_maskradius_integrated.pdf}
% 	\caption[Classificação por morfologia]
% 	{Valores integrados das mesmas propriedades da Fig.\ \ref{fig:histosample} para as 184
% galáxias da amostra, separadas em classes morfológicas. No primeiro painel, temos o número de
% galáxias dentro de cada classe morfológica. Cada caixa tem altura definida pelo primeiro e terceiro
% quartil da distribuição dentro de um tipo morfológico. Uma faixa preta marca a mediana e uma
% estrela a média. Em cada caixa, a linha pontilhada vertical se estende mostrando o intervalo de
% $3\sigma$. Valores que ficam fora do intervalo de $3\sigma$ são marcados por uma cruz vermelha.}
% 	\label{fig:amostraMorf}
% \end{figure}
%
% Estamos em fase de finalização de um artigo em que comparamos a relação entre a taxa de formação
% estelar e a massa para diferentes classes morfológicas. Esse artigo já foi submetido e deve sair
% logo agora no início de 2016.
%
%
%
%
%
%
% \section{Perfis radiais}
% \label{sec:amostra:rad}
%
% Uma maneira interessante de analisar galáxias é produzir perfis radiais para as propriedades
% físicas. Esse tipo de média azimutal (tanto em classes definidas por anéis circulares quanto em
% anéis elípticos) diminui o espalhamento dos pontos. Para a análise individual de cada galáxia também
% permite estudo da evolução das propriedades ao longo do raio. Quando colocamos todas as galáxias na
% mesma análise, a vantagem dos {\em bins} radiais vem do balanceamento da influência de cada galáxia
% quando analisamos todas juntas. Para que seja possível este ``empilhamento'' de galáxias, estas
% médias radiais são feitas definindo-se um raio efetivo para cada galáxia. No nosso caso utilizamos
% como raio efetivo aquele que comporta metade da luz da galáxia (HLR) e definimos 30 anéis com
% espessura de 0.1 HLR (ou seja, indo até 3 HLR) partindo do pixel central de cada galáxia.
%
% No artigo de \citet{GonzalezDelgado.etal.2014a} os autores discutem as estruturas radiais de algumas
% propriedades estelares, aplicando este tipo de estudo para 107 galáxias no CALIFA. Nele são
% derivados os raios que contém metade da luz (HLR) e metade da massa ({\em half-mass radius} - HMR) e
% deste resultado concluem que as galáxias são em geral 15\% mais compactas em massa do que em luz.
% Também mostram que algumas propriedades, como idade estelar média, extinção por poeira e densidade
% superficial de massa estelar são bem representados pelo seus valores medidas a 1 HLR.
%
% Escolhemos utilizar perfis radiais em anéis elípticos neste trabalho, calculando a média entre todas
% as zonas não mascaradas dentro de cana anel em cada galáxia. Como um exemplo, podemos observar na
% Fig.\ \ref{fig:K0140xYRadProf} três exemplos de mapas e perfis radiais ($x_Y$, $\tauVS$ e $\tauVN$)
% da galáxia NGC1667 (objeto CALIFA 140). Em destaque (azul) temos o valor integrado para a galáxia.
% Dentro de nosso trabalho utilizamos as medidas em zonas, em perfis radiais e quando necessário,
% integradas (resolvendo para o disco ou para a galáxia completa), nos possibilitando portanto
% verificar diferenças nestes tipos de abordagens.
%
% \begin{figure}
% 	\centering
% 	\includegraphics[width=0.99\textwidth]{figuras/K0140_xY_radialProfile_realsample.pdf}
% 	\caption[Imagem e exemplos de mapas e perfis radiais]
% 	{Imagem do \SDSS da galáxia NGC1667 (CALIFA 140). Em cada fileira aparece o mapa e o perfil radial
% da fração de luz proveniente das populações jovens ($x_Y$ - \emph{primeira fileira}), do
% coeficiente de extinção resultante da síntese de populações estelares ($\tauVS$ - \emph{segunda
% fileira}) e do coeficiente de extinção por decremento de Balmer ($\tauVN$ - \emph{terceira
% fileira}). Nos mapas duas elipses concêntricas marcam 1 e 2 HLR. Em cada gráfico do perfil radial
% aparece no fundo em cinza os valores para as zonas, em linha tracejada preta a mediana da
% distribuição ao longo do raio e em azul tracejado o valor integrado para a galáxia, além do perfil
% radial (linha preta contínua).}
% 	\label{fig:K0140xYRadProf}
% \end{figure}
%
% O perfil radial médio das principais propriedades utilizadas neste trabalho podem ser vistas na
% Fig.\ \ref{fig:RadProfProps} juntamente com seus histogramas na Fig. \ref{fig:HistoRadProfProps}.
% Note que estamos apenas colocando os pontoa onde o reaio esteja entre 0.7 e 3 HLR. Elas são muito
% importantes em todos os aspectos que abordaremos: formação estelar, poeira e a conversão de
% densidade superficial de poeira em densidade superficial de gás em discos de galáxias espirais.
% Podemos ver que existem alguns gradientes negativos (crescem de fora para dentro) bem definidos de
% densidade superficial da taxa de formação estelar ($\Sigma_{\mathrm{SFR}}$), a metalicidade nebular,
% ($\log O/H$), densidade superficial de massa estelar ($\mu_\star$). Outras propriedades parecem não
% variar muito dentro do disco. Alguns perfis mudam de tendência ao passar de 2 HLR, mas vale
% ressaltar que na maioria das distribuições, os pontos acima de 2 HLR ultrapassam $1\sigma$ da
% distribuição. Embora não seja tão forte, existe um gradiente positivo na fração em luz de populações
% jovens.
%
% \begin{figure}
% 	\centering
% 	\includegraphics[width=0.9\textwidth]{figuras/props_R.pdf}
% 	\caption[Perfis radiais das propriedades físicas]
% 	{Perfis radiais médios (entre 0.7 e 3 HLR) das principais propriedades físicas abordadas neste
% trabalho. Densidade superficial da taxa de formação estelar detectada por \Halpha e pela síntese
% ({\em painéis a e b}), idade média das populações estelares ({\em painel c}), metalicidade nebular
% ({\em painel d}), fração em luz de populações jovens ({\em painel e}), densidade superficial de
% massa estelar ({\em painel f}), metalicidade média das populações estelares ({\em painel g}),
% coeficiente de extinção da síntese e do decremento de Balmer ({\em painéis h e i}). Em cada painel
% vemos também os contornos definindo $1\sigma$, $2\sigma$ e $3\sigma$ da distribuição. As linhas
% marcam a mediana (linha contínua) e os 5, 16, 64, 95 percentis (linhas tracejadas).}
% 	\label{fig:RadProfProps}
% \end{figure}
%
% \begin{figure}
% 	\centering
% 	\includegraphics[width=0.99\textwidth]{figuras/histo_props_R.pdf}
% 	\caption[Histogramas dos perfis radiais das propriedades físicas]
% 	{Histogramas normalizados de todas as propriedades físicas da Fig.\ \ref{fig:RadProfProps}. Os
% valores no canto superior direito marcam a média, mediana, desvio padrão, máximo e mínimo das
% distribuições.}
% 	\label{fig:HistoRadProfProps}
% \end{figure}

% Figuras:
% - histograma tipos - histograma massa - influências dos cortes em tauV, raio e x_Y - Anexo: Lista
% de galáxias com massa, redshift, idade, etc ...
%% End of this chapter
